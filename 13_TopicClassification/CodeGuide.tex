% Options for packages loaded elsewhere
\PassOptionsToPackage{unicode}{hyperref}
\PassOptionsToPackage{hyphens}{url}
\PassOptionsToPackage{dvipsnames,svgnames,x11names}{xcolor}
%
\documentclass[
  letterpaper,
  DIV=11,
  numbers=noendperiod]{scrartcl}

\usepackage{amsmath,amssymb}
\usepackage{iftex}
\ifPDFTeX
  \usepackage[T1]{fontenc}
  \usepackage[utf8]{inputenc}
  \usepackage{textcomp} % provide euro and other symbols
\else % if luatex or xetex
  \usepackage{unicode-math}
  \defaultfontfeatures{Scale=MatchLowercase}
  \defaultfontfeatures[\rmfamily]{Ligatures=TeX,Scale=1}
\fi
\usepackage{lmodern}
\ifPDFTeX\else  
    % xetex/luatex font selection
\fi
% Use upquote if available, for straight quotes in verbatim environments
\IfFileExists{upquote.sty}{\usepackage{upquote}}{}
\IfFileExists{microtype.sty}{% use microtype if available
  \usepackage[]{microtype}
  \UseMicrotypeSet[protrusion]{basicmath} % disable protrusion for tt fonts
}{}
\makeatletter
\@ifundefined{KOMAClassName}{% if non-KOMA class
  \IfFileExists{parskip.sty}{%
    \usepackage{parskip}
  }{% else
    \setlength{\parindent}{0pt}
    \setlength{\parskip}{6pt plus 2pt minus 1pt}}
}{% if KOMA class
  \KOMAoptions{parskip=half}}
\makeatother
\usepackage{xcolor}
\usepackage[tmargin=1cm,bmargin=1cm]{geometry}
\setlength{\emergencystretch}{3em} % prevent overfull lines
\setcounter{secnumdepth}{-\maxdimen} % remove section numbering
% Make \paragraph and \subparagraph free-standing
\ifx\paragraph\undefined\else
  \let\oldparagraph\paragraph
  \renewcommand{\paragraph}[1]{\oldparagraph{#1}\mbox{}}
\fi
\ifx\subparagraph\undefined\else
  \let\oldsubparagraph\subparagraph
  \renewcommand{\subparagraph}[1]{\oldsubparagraph{#1}\mbox{}}
\fi


\providecommand{\tightlist}{%
  \setlength{\itemsep}{0pt}\setlength{\parskip}{0pt}}\usepackage{longtable,booktabs,array}
\usepackage{calc} % for calculating minipage widths
% Correct order of tables after \paragraph or \subparagraph
\usepackage{etoolbox}
\makeatletter
\patchcmd\longtable{\par}{\if@noskipsec\mbox{}\fi\par}{}{}
\makeatother
% Allow footnotes in longtable head/foot
\IfFileExists{footnotehyper.sty}{\usepackage{footnotehyper}}{\usepackage{footnote}}
\makesavenoteenv{longtable}
\usepackage{graphicx}
\makeatletter
\def\maxwidth{\ifdim\Gin@nat@width>\linewidth\linewidth\else\Gin@nat@width\fi}
\def\maxheight{\ifdim\Gin@nat@height>\textheight\textheight\else\Gin@nat@height\fi}
\makeatother
% Scale images if necessary, so that they will not overflow the page
% margins by default, and it is still possible to overwrite the defaults
% using explicit options in \includegraphics[width, height, ...]{}
\setkeys{Gin}{width=\maxwidth,height=\maxheight,keepaspectratio}
% Set default figure placement to htbp
\makeatletter
\def\fps@figure{htbp}
\makeatother

\KOMAoption{captions}{tableheading}
\makeatletter
\makeatother
\makeatletter
\makeatother
\makeatletter
\@ifpackageloaded{caption}{}{\usepackage{caption}}
\AtBeginDocument{%
\ifdefined\contentsname
  \renewcommand*\contentsname{Table of contents}
\else
  \newcommand\contentsname{Table of contents}
\fi
\ifdefined\listfigurename
  \renewcommand*\listfigurename{List of Figures}
\else
  \newcommand\listfigurename{List of Figures}
\fi
\ifdefined\listtablename
  \renewcommand*\listtablename{List of Tables}
\else
  \newcommand\listtablename{List of Tables}
\fi
\ifdefined\figurename
  \renewcommand*\figurename{Figure}
\else
  \newcommand\figurename{Figure}
\fi
\ifdefined\tablename
  \renewcommand*\tablename{Table}
\else
  \newcommand\tablename{Table}
\fi
}
\@ifpackageloaded{float}{}{\usepackage{float}}
\floatstyle{ruled}
\@ifundefined{c@chapter}{\newfloat{codelisting}{h}{lop}}{\newfloat{codelisting}{h}{lop}[chapter]}
\floatname{codelisting}{Listing}
\newcommand*\listoflistings{\listof{codelisting}{List of Listings}}
\makeatother
\makeatletter
\@ifpackageloaded{caption}{}{\usepackage{caption}}
\@ifpackageloaded{subcaption}{}{\usepackage{subcaption}}
\makeatother
\makeatletter
\@ifpackageloaded{tcolorbox}{}{\usepackage[skins,breakable]{tcolorbox}}
\makeatother
\makeatletter
\@ifundefined{shadecolor}{\definecolor{shadecolor}{rgb}{.97, .97, .97}}
\makeatother
\makeatletter
\makeatother
\makeatletter
\makeatother
\ifLuaTeX
  \usepackage{selnolig}  % disable illegal ligatures
\fi
\IfFileExists{bookmark.sty}{\usepackage{bookmark}}{\usepackage{hyperref}}
\IfFileExists{xurl.sty}{\usepackage{xurl}}{} % add URL line breaks if available
\urlstyle{same} % disable monospaced font for URLs
\hypersetup{
  colorlinks=true,
  linkcolor={blue},
  filecolor={Maroon},
  citecolor={Blue},
  urlcolor={Blue},
  pdfcreator={LaTeX via pandoc}}

\author{}
\date{}

\begin{document}
\ifdefined\Shaded\renewenvironment{Shaded}{\begin{tcolorbox}[frame hidden, interior hidden, enhanced, breakable, sharp corners, borderline west={3pt}{0pt}{shadecolor}, boxrule=0pt]}{\end{tcolorbox}}\fi

Mistral (8GB) was prompted to classify 9,994 comments into the following
16 categories (relevant/irrelevant). Each topic was categorized
separately, resulting in a total of \(16 \times 9,994 = 159,904\)
prompts. For 4 of the topics, we hypothesized that mostly Republicans or
conservative would be triggered. We aim to manually code 10\% of the
comments of the 9,994 to assess the accuracy of the LLM as a classifier.

\begin{longtable}[]{@{}ccc@{}}
\toprule\noalign{}
& Topics triggering Cons/Reps & Topics triggering Libs/Dems \\
\midrule\noalign{}
\endhead
\bottomrule\noalign{}
\endlastfoot
1 & Anti Cons/Reps Rhetoric & Anti Libs/Dems Rhetoic \\
2 & Pro abortion & Anti abortion \\
3 & Anti gun & Pro gun \\
4 & Pro immigration & Anti immigration \\
5 & Pro Hillary Clinton & Anti Hillary Clinton \\
6 & Anti Donald Trump & Pro Donald Trump \\
7 & Drugs & - \\
8 & Suggestive content & - \\
9 & Anti American Rhetoric & - \\
10 & Anti Christianity & - \\
\end{longtable}

\clearpage

\hypertarget{drugs}{%
\subparagraph{1 Drugs}\label{drugs}}

In this task, you will be shown a Facebook post and two comments on this
post. One comment is from person A and the other comment is from person
B.

Your job is to determine if the comment by B is related to the
consumption, production, or distribution of drugs. Drugs refer to
substances with psychoactive effects.

Your answer must be a single number: 0 if irrelevant, 1 if relevant.

Here are some examples with reasoning. Use the following format for your
output:

Example 1: Comment by B: ``She is high on drugs of stupidity!'' Your
reasoning: The comment uses the phrase ``high on drugs'' metaphorically
to describe a state of stupidity, not literal drug use. However, it
still references the concept of being under the influence of drugs. Your
answer: 1

Example 2: Comment by B: ``And who's fault is it that the cartels exist
?? Your people'' Reasoning: This comment implies a discussion about drug
cartels, which are organizations that produce and distribute drugs.
Therefore, it is relevant to the topic of drugs. Your answer: 1

Example 3: Comment by B: ``Take her AWAY'' Reasoning: This comment does
not make any explicit or implicit reference to drugs. Hence, the comment
is irrelevant to the topic of drugs. Your answer: 0

\hypertarget{suggestive-content}{%
\subparagraph{2 Suggestive content}\label{suggestive-content}}

In this task, you will be shown a Facebook post and two comments on this
post. One comment is from person A and the other comment is from person
B.

Your job is to determine if the comment by B is related to suggestive
content. Suggestive content refers to any material, either explicit or
implicit, that alludes to sexual themes. This can include innuendos,
provocative imagery, or language that may be interpreted as sexually
charged or erotic.

Your answer must be a single number: 0 if irrelevant, 1 if relevant.

Here are some examples with reasoning. Use the following format for your
output:

Example 1: Comment by B: ``If those prostitutes have balls.'' Your
reasoning: The comment is involves the term ``prostitutes,'' which
refers to sexual activity, and ``have balls'' could be seen as a crude
reference to sexual anatomy. Hence, it is relevant to suggestive
content. Your answer: 1

Example 2: Comment by B: ``They have hunting\ldots{} And dicks on their
brain!'' Reasoning: The term ``dicks'' is a sexual reference. Thus, this
comment involves suggestive content. Your answer: 1

Example 3: Comment by B: ``Very painful to watch.'' Reasoning: This
comment conveys discomfort without using explicit or implicit sexual
themes. Your answer: 0

\hypertarget{negative-portrayal-of-christianity}{%
\subparagraph{3 Negative portrayal of
Christianity}\label{negative-portrayal-of-christianity}}

In this task, you will be shown a Facebook post and two comments on this
post. One comment is from person A and the other comment is from person
B.

Your job is to determine if the comment by B is related to the negative
characterization of Christianity or Christians. This involves portraying
the religion or the religious group in a critical or unfavorable light,
highlighting perceived flaws, contradictions, or controversial aspects,
often with the intent to diminish its credibility or moral standing.

Your answer must be a single number: 0 if irrelevant, 1 if relevant.

Here are some examples with reasoning. Use the following format for your
output:

Example 1: Comment by B: ``Another good `Christian' pedophile!'' Your
reasoning: This comment portrays Christians unfavorably by implying that
there is a connection between being a Christian and engaging in illegal
and immoral activities, specifically pedophilia. Your answer: 1

Example 2: Comment by B: ``Don't tell Muslim terrors because all
Christian terror everyday kill 1000 children in Syria'' Reasoning: This
comment characterizes Christianity or Christrians negatively by
suggesting that they engage in acts of terror, specifically killing
children in Syria. Your answer: 1

Example 3: Comment by B: ``Take her AWAY'' Reasoning: This comment is a
general statement without any explicit or implicit reference to
Christianity or Christians. Your answer: 0

\hypertarget{negative-portrayal-of-america}{%
\subparagraph{4 Negative Portrayal of
America}\label{negative-portrayal-of-america}}

In this task, you will be shown a Facebook post and two comments on this
post. One comment is from person A and the other comment is from person
B.

Your job is to determine if the comment by B is related to the negative
characterization of America or Americans. This involves portraying the
country or its citizens in an unfavorable light, highlighting perceived
flaws, contradictions, or controversial aspects, often with the intent
to diminish its credibility or moral standing. Your answer must be a
single number: 0 if irrelevant, 1 if relevant.

Here are some examples with reasoning. Use the following format for your
output:

Example 1: Comment by B: ``You mean unlike the American killers of
babies, rapist, or serial killers?'' Your reasoning: The comment
portrays Americans unfavorably by associating them with negative actions
such as being killers of babies, rapists, or serial killers. Your
answer: 1

Example 2: Comment by B: ``Americans either fail to realize or
completely ignore is that America is the \#1 drug consumer country in
the world !'' Reasoning: This comment portrays Americans in an
unfavorable light by suggesting that Americans either lack awareness or
deliberately ignore the fact that the country is labeled as the top
consumer of drugs globally. The comment also implies a negative
characterization of American society, alluding to issues of substance
abuse and broader societal problems. Your answer: 1

Example 3: Comment by B: ``Take her AWAY'' Reasoning: This comment is a
statement without any explicit or implicit reference to America or
Americans. Your answer: 0

\hypertarget{anti-trump}{%
\subparagraph{5 Anti-Trump}\label{anti-trump}}

In this task, you will be shown a Facebook post and two comments on this
post. One comment is from person A and the other comment is from person
B.

Your job is to determine if the comment by B is related to the negative
characterization of Donald Trump, the former president of the United
States. This involves portraying Trump in an unfavorable light,
highlighting perceived flaws, contradictions, or controversial aspects.
Your answer must be a single number: 0 if irrelevant, 1 if relevant.

Here are some examples with reasoning. Use the following format for your
output:

Example 1: Comment by B: ``Trump is a friggen rapist!'' Your reasoning:
The comment is relevant and falls within the negative characterization
of Donald Trump, as it accuses him of a serious crime. Your answer: 1

Example 2: Comment by B: ``Can Trump shut up for atleast one week. Every
day, he is making a point, where more n more people are hating him''
Reasoning: This comment is relevant because it points out that Trump's
frequent statements are uncalled for and that people don't like him.
Your answer: 1

Example 3: Comment by B: ``So tired of this progressive rant. If you
come here illegally you need to go home. PERIOD! Get them the hell out
of America!'' Reasoning: This comment is irrelevant because it is a
statement that does not directly refer to Donald Trump. Your answer: 0

\hypertarget{pro-trump}{%
\subparagraph{6 Pro-Trump}\label{pro-trump}}

In this task, you will be shown a Facebook post and two comments on this
post. One comment is from person A and the other comment is from person
B.

Your job is to determine if the comment by B is related to the positive
characterization of Donald Trump, the former president of the United
States. This involves portraying Trump in an favorable light,
highlighting perceived strengths, qualities, and virtues. Your answer
must be a single number: 0 if irrelevant, 1 if relevant.

Here are some examples with reasoning. Use the following format for your
output:

Example 1: Comment by B: ``Vote for Trump! \#Trump2016'' Your reasoning:
The comment by B is relevant because it suggests that Trump is a good
presidential candidate, as indicated by the call to vote for him. Your
answer: 1

Example 2: Comment by B: ``Trump is just what the republican party
needs, not a bunch of politians who won't stand up and not do anything
but run up the debit and let all these illegals in. Everything he said
was right. Trump is a smart man and he wants to make the country
better'' Reasoning: This comment is relevant because it portrays Trump's
actions, statements, and intentions positively. Your answer: 1

Example 3: Comment by B: ``If Hillary is elected president, Putin said
he will wage nuclear war against the United States, but I guess that's
not a big deal\ldots{}'' Reasoning: This comment is irrelevant because
it is a statement that does not directly refer to Donald Trump. Your
answer: 0

\hypertarget{anti-clinton}{%
\subparagraph{7 Anti-Clinton}\label{anti-clinton}}

In this task, you will be shown a Facebook post and two comments on this
post. One comment is from person A and the other comment is from person
B.

Your job is to determine if the comment by B is related to the negative
characterization of Hillary Clinton, the Democratic Party's 2016 nominee
for president of the United States. This involves portraying Hillary
Clinton in an unfavorable light, highlighting perceived flaws,
contradictions, or controversial aspects. Your answer must be a single
number: 0 if irrelevant, 1 if relevant.

Here are some examples with reasoning. Use the following format for your
output:

Example 1: Comment by B: ``Yes WE NEED TO LOCK HILLARY AND HER FRIENDS
UP!'' Your reasoning: The comment is relevant because it states that
Hillary Clinton and her friends should be incarcerated, suggesting
illegal action on their part. Your answer: 1

Example 2: Comment by B: ``Hillary is a big fat liar and she will never
be president.'' Reasoning: The comment is relevant because it depicts
Hillary Clinton negatively, accusing her of dishonesty and implying she
has no shot at the presidency even if she attempts it. Your answer: 1

Example 3: Comment by B: ``Trump is just what the republican party
needs, not a bunch of politians who won't stand up and not do anything
but run up the debit and let all these illegals in.'' Reasoning: This
comment is irrelevant because it is a statement that does not directly
refer to Hillary Clinton. Your answer: 0

\hypertarget{pro-hillary}{%
\subparagraph{8 Pro-Hillary}\label{pro-hillary}}

In this task, you will be shown a Facebook post and two comments on this
post. One comment is from person A and the other comment is from person
B.

Your job is to determine if the comment by B is related to the positive
characterization of Hillary Clinton, the Democratic Party's 2016 nominee
for president of the United States. This involves portraying Hillary
Clinton in an favorable light, highlighting perceived strengths,
qualities, and virtues. Your answer must be a single number: 0 if
irrelevant, 1 if relevant.

Here are some examples with reasoning. Use the following format for your
output:

Example 1: Comment by B: ``Recount!! Go girl! Hillary for president!!''
Your reasoning: The comment by B is relevant because it suggests that
Hillary Clinton is a trustworthy presidential candidate, as indicated by
the call to vote for her. Your answer: 1

Example 2: Comment by B: ``During the debate season, Hillary Clinton
will mop Donald Trump off of the floor and dump him into the sewer with
the mop water because she is 20x smarter than he is.'' Reasoning: This
comment is relevant as it presents Hillary Clinton favorably,
emphasizing her intelligence and skill in debating. Your answer: 1

Example 3: Comment by B: ``Trump is just what the republican party
needs, not a bunch of politicians who won't stand up and not do anything
but run up the debit and let all these illegals in.'' Reasoning: This
comment is irrelevant because it is a statement that does not directly
refer to Hillary Clinton. Your answer: 0

\hypertarget{anti-immigrant}{%
\subparagraph{9 Anti-Immigrant}\label{anti-immigrant}}

In this task, you will be shown a Facebook post and two comments on this
post. One comment is from person A and the other comment is from person
B.

Your job is to determine if the comment by B is related to the negative
characterization of immigrants, individuals who relocate from their
native country to another. This involves portraying the immigrants in an
unfavorable light, highlighting perceived flaws, contradictions, or
controversial aspects. Your answer must be a single number: 0 if
irrelevant, 1 if relevant.

Here are some examples with reasoning. Use the following format for your
output:

Example 1: Comment by B: ``The immigrants steal our jobs\ldots. even our
speeches!'' Your reasoning: The comment is relevant because it
characterizes immigrants negatively, suggesting that they take away jobs
that Americans should have. Your answer: 1

Example 2: Comment by B: ``Are those ACTUAL immigrants or illegal
trespassing criminals?'' Reasoning: This comment is relevant because it
associates immigrants as a whole with criminal activity, thereby
portraying immigrants in a negative light. Your answer: 1

Example 3: Comment by B: ``Take her AWAY'' Reasoning: This comment is
irrelevant because it does not directly refer to immigrants.

\hypertarget{pro-immigrants}{%
\subparagraph{10 Pro-immigrants}\label{pro-immigrants}}

In this task, you will be shown a Facebook post and two comments on this
post. One comment is from person A and the other comment is from person
B.

Your job is to determine if the comment by B is related to the positive
characterization of immigrants, individuals who relocate from their
native country to another. This involves portraying the immigrants in a
favorable light, highlighting perceived strengths, qualities, and
virtues. Your answer must be a single number: 0 if irrelevant, 1 if
relevant.

Here are some examples with reasoning. Use the following format for your
output:

Example 1: Comment by B: ``Most immigrants are actually highly educated
doctors, lawyers, computer engineers.'' Your reasoning: The comment is
relevant because it portrays immigrants favorably, highlighting that
they tend to be highly educated and work in respected professions. Your
answer: 1

Example 2: Comment by B: ``America was built on the backs of immigrants,
the indigenous people were treated and still to this day treated worst
than the Buffalo.'' Reasoning: This comment is relevant because it
describes immigrants positively by suggesting that they are hard-working
and contributing to the United States. Your answer: 1

Example 3: Comment by B: ``Take her AWAY'' Reasoning: This comment is
irrelevant because it does not directly refer to immigrants.

\hypertarget{anti-abortion}{%
\subparagraph{11 Anti-abortion}\label{anti-abortion}}

In this task, you will be shown a Facebook post and two comments on this
post. One comment is from person A and the other comment is from person
B.

Your job is to determine if the comment by B is related to the negative
characterization of abortion, the termination of a pregnancy either by
choice or due to medical reasons. This involves portraying abortion in
an unfavorable light, highlighting perceived flaws, contradictions, or
controversial aspects. Your answer must be a single number: 0 if
irrelevant, 1 if relevant.

Here are some examples with reasoning. Use the following format for your
output:

Example 1: Comment by B: ``All abortions are murders!'' Your reasoning:
The comment is relevant because it characterizes abortion negatively,
suggesting that it equates to taking a human life. Your answer: 1

Example 2: Comment by B: ``Every woman who wants an abortion should see
this. Children are truly gifts from God!'' Reasoning: This comment is
relevant because it states that children are a gift from god, thereby
implying a stance against abortion. Your answer: 1

Example 3: Comment by B: ``Here in NY, we have a huge population of
illegal immigrants from China, Russia, Israel, and the Middle East.''
Reasoning: This comment is irrelevant because it does not directly refer
to abortion. Your answer: 0

\hypertarget{pro-abortion}{%
\subparagraph{12 Pro-abortion}\label{pro-abortion}}

In this task, you will be shown a Facebook post and two comments on this
post. One comment is from person A and the other comment is from person
B.

Your job is to determine if the comment by B is related to the positive
characterization of abortion, the termination of a pregnancy either by
choice or due to medical reasons. This involves portraying abortion in a
favorable light, highlighting perceived benefits or advantages. Your
answer must be a single number: 0 if irrelevant, 1 if relevant.

Here are some examples with reasoning. Use the following format for your
output:

Example 1: Comment by B: ``Many women are unintentionally impregnated by
a RAPIST. Involuntarily having sex is what rape is, if you didn't catch
that. A woman should be able to choose if she wants to have an
abortion'' Your reasoning: The comment is relevant because it
characterizes abortion positively, arguing that abortion should be a
necessary option for women who become pregnant as a result of rape. Your
answer: 1

Example 2: Comment by B: ``Some women don't want to have kids at a
particular time. Some never want to have kids. Second, most women who
have abortions do so because of financial reasons. Third, contraception
isn't 100 percent effective. Some types are only 75 percent or so
effective.'' Reasoning: This comment is relevant because it views
abortion positively, emphasizing that a woman should be able to decide
when and under what circumstances to have kids. Your answer: 1

Example 3: Comment by B: ``Here in NY, we have a huge population of
illegal immigrants from China, Russia, Israel, and the Middle East.''
Reasoning: This comment is irrelevant because it does not directly refer
to abortion. Your answer: 0

\hypertarget{anti-gun}{%
\subparagraph{13 Anti-gun}\label{anti-gun}}

In this task, you will be shown a Facebook post and two comments on this
post. One comment is from person A and the other comment is from person
B.

Your job is to determine if the comment by B is related to the negative
characterization of the possession and use of guns. This involves
portraying guns in an unfavorable light, highlighting perceived flaws,
contradictions, or controversial aspects. Your answer must be a single
number: 0 if irrelevant, 1 if relevant.

Here are some examples with reasoning. Use the following format for your
output:

Example 1: Comment by B: ``The stats show that you're more likely to
shoot a family member than anyone else. Guns don't kill people, but
people with guns can and do kill more people than those without. Who
would have been at fault if the kids shot out the car window and killed
someone?'' Your reasoning: The comment is relevant because it
characterizes guns negatively, arguing that they result unintended
consequences such as accidental killings. Your answer: 1

Example 2: Comment by B: ``We have too many guns, too easily had, by too
many people. Guns within easy reach are far more likely to do harm than
good.'' Reasoning: This comment is relevant because it characterizes
guns negatively, highlighting their excessive presence and harmful
effects. Your answer: 1

Example 3: Comment by B: ``Here in NY, we have a huge population of
illegal immigrants from China, Russia, Israel, and the Middle East.''
Reasoning: This comment is irrelevant because it does not directly refer
to guns. Your answer: 0

\hypertarget{pro-gun}{%
\subparagraph{14 Pro-gun}\label{pro-gun}}

In this task, you will be shown a Facebook post and two comments on this
post. One comment is from person A and the other comment is from person
B.

Your job is to determine if the comment by B is related to the positive
characterization of the possession and use of guns. This involves
portraying guns in a favorable light, highlighting perceived benefits or
advantages. Your answer must be a single number: 0 if irrelevant, 1 if
relevant.

Here are some examples with reasoning. Use the following format for your
output:

Example 1: Comment by B: ``This is exactly what the government wants,
take our guns so they can declare the U.S. a dictatorship!'' Your
reasoning: The comment is relevant because it portrays guns in a
positive light, arguing that they are necessary for preventing
government overreach. Your answer: 1

Example 2: Comment by B: ``Women are one of the fastest group of gun
owners in the country. Many single mothers and single women carry
weapons, it is the best defense.'' Reasoning: This comment is relevant
because it views guns positively, emphasizing that they are essential to
protection. Your answer: 1

Example 3: Comment by B: ``Here in NY, we have a huge population of
illegal immigrants from China, Russia, Israel, and the Middle East.''
Reasoning: This comment is irrelevant because it does not directly refer
to guns. Your answer: 0

\hypertarget{anti-conservative-or-anti-republican-rhetoric}{%
\subparagraph{15 Anti-Conservative or Anti-Republican
Rhetoric}\label{anti-conservative-or-anti-republican-rhetoric}}

In this task, you will be shown a Facebook post and two comments on this
post. One comment is from person A and the other comment is from person
B.

Your job is to determine if the comment by B is related to the negative
characterization of Republicans or conservatives. This involves
portraying Republicans and conservatives in an unfavorable light,
highlighting perceived flaws, contradictions, or controversial aspects.
Your answer must be a single number: 0 if irrelevant, 1 if relevant.

Here are some examples with reasoning. Use the following format for your
output:

Example 1: Comment by B: ``Conservatives can also be whiny.'' Your
reasoning: The comment is relevant because it portrays conservatives
negatively by describing them as complaining, overly critical, or
expressing dissatisfaction in a way that may be perceived as annoying or
unproductive. Your answer: 1

Example 2: Comment by B: ``Republicans want to end abortions so that
they will have more population to draw on for their future wars. Think
about it: End abortions, brings on many children that the parents can't
care for (likely because of finances).'' Reasoning: This comment is
relevant because it characterizes Republicans negatively by suggesting
that their opposition to abortion is driven by a desire to increase the
population for potential future wars, implying a lack of concern for the
welfare of children and families. Your answer: 1

Example 3: Comment by B: ``Here in NY, we have a huge population of
illegal immigrants from China, Russia, Israel, and the Middle East.''
Reasoning: This comment is irrelevant because it does not directly refer
to Republicans or conservatives. Your answer: 0

\hypertarget{anti-liberal-or-anti-democrat-rhetoric}{%
\subparagraph{16 Anti-Liberal or Anti-Democrat
Rhetoric}\label{anti-liberal-or-anti-democrat-rhetoric}}

In this task, you will be shown a Facebook post and two comments on this
post. One comment is from person A and the other comment is from person
B.

Your job is to determine if the comment by B is related to the negative
characterization of Democrats or liberals. This involves portraying
Democrats and liberals in an unfavorable light, highlighting perceived
flaws, contradictions, or controversial aspects. Your answer must be a
single number: 0 if irrelevant, 1 if relevant.

Here are some examples with reasoning. Use the following format for your
output:

Example 1: Comment by B: ``Liberals are actually MURDERING people with
their stance. The liberals are the real murderers.'' Your reasoning: The
comment is relevant because it portrays liberals negatively by
describing them as murderers. Your answer: 1

Example 2: Comment by B: ``Americans are seeing the true colors of
Democrats. Most are trash that don't pay taxes. They live off the
taxpayers!!'' Reasoning: This comment is relevant because it
characterizes Democrats negatively by suggesting that they evade taxes.
Your answer: 1

Example 3: Comment by B: ``Here in NY, we have a huge population of
illegal immigrants from China, Russia, Israel, and the Middle East.''
Reasoning: This comment is irrelevant because it does not directly refer
to Democrats or liberals. Your answer: 0

\hypertarget{negative-characterization-of-white-people}{%
\subparagraph{17 Negative characterization of White
people}\label{negative-characterization-of-white-people}}

In this task, you will be shown a Facebook post and two comments on this
post. One comment is from person A and the other comment is from person
B.

Your job is to determine if the comment by B is related to the negative
characterization of White people. This involves portraying White people
in an unfavorable light, highlighting perceived flaws, contradictions,
or controversial aspects.

Your answer must be a single number: 0 if irrelevant, 1 if relevant.

Here are some examples with reasoning. Use the following format for your
output:

Example 1: Comment by B: ``White trash should lay low or move to another
state.'' Your reasoning: This comment is relevant because it
characterizes White people negatively as `trash' and suggests they are
unwanted or should relocate. Your answer: 1

Example 2: Comment by B: ``Almost all Caucasians have ancestors that
were thieves, murderers, and rapists'' Reasoning: This comment is
relevant because it characterizes White people negatively by associating
them with criminal behavior. Your answer: 1

Example 3: Comment by B: ``Take her AWAY'' Reasoning: This comment is
irrelevant because it does not directly refer to White people. Your
answer: 0

\hypertarget{killing}{%
\subparagraph{18 Killing}\label{killing}}

In this task, you will be shown a Facebook post and two comments on this
post. One comment is from person A and the other comment is from person
B.

Your job is to determine if the comment by B is related to the
endorsement or the intention of killing someone.

Your answer must be a single number: 0 if irrelevant, 1 if relevant.

Here are some examples with reasoning. Use the following format for your
output:

Example 1: Comment by B: ``Let's nuke them now!'' Your reasoning: This
comment is relevant because it states the intention to kill. Your
answer: 1

Example 2: Comment by B: ``Execute them! That will teach them a
lesson!'' Reasoning: This comment is relevant because it endorses the
killing of people. Your answer: 1

Example 3: Comment by B: ``Take her AWAY'' Reasoning: This comment is
irrelevant because it does not directly refer to the endorsement or the
intention of killing someone. Your answer: 0

\hypertarget{negative-characterization-of-barack-or-michelle-obama}{%
\subparagraph{19 Negative characterization of Barack or Michelle
Obama}\label{negative-characterization-of-barack-or-michelle-obama}}

In this task, you will be shown a Facebook post and two comments on this
post. One comment is from person A and the other comment is from person
B.

Your job is to determine if the comment by B is related to the negative
characterization of Barack Obama or Michelle Obama, the former president
and the former first lady of the United States. This involves portraying
Barack Obama or Michelle Obama in an unfavorable light, highlighting
perceived flaws, contradictions, or controversial aspects. Your answer
must be a single number: 0 if irrelevant, 1 if relevant.

Here are some examples with reasoning. Use the following format for your
output:

Example 1: Comment by B: ``Barack Obama needs to be tried for treason''
Your reasoning: The comment is relevant because it characterizes Barack
Obama negatively, accusing him of treason and advocates for legal action
against him. Your answer: 1

Example 2: Comment by B: ``Anyone would be a 1000 time better first lady
than Michelle Obama!'' Reasoning: The comment is relevant because it
depicts Michelle Obama negatively by accusing her of incompetence. Your
answer: 1

Example 3: Comment by B: ``Take her AWAY'' Reasoning: This comment is
irrelevant because it is a statement that does not directly refer to
Barack Obama or Michelle Obama. Your answer: 0



\end{document}
