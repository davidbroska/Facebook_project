\documentclass{article}
\usepackage{graphicx} % Required for inserting images
\usepackage{authblk}
\usepackage{cite}
\usepackage{natbib}
\usepackage{dcolumn}




\date{Draft -- please do not circulate.}

\title{(In)civility on Display: Social Media Discourse through the Ideological Lens}

\author[1]{David Broska}
\author[2]{Byungkyu Lee}
\author[3]{Barum Park}
\author[4]{Daniel A. McFarland}
\affil[1,4]{Stanford University}
\affil[2]{New York University}
\affil[2]{Cornell University}

\begin{document}

\maketitle

\section{Introduction}

Conversations are at the heart of democracies. Discussions of everyday and fundamental issues in parliaments, media, and kitchen tables inform the political process. An exchange based on reason rather than content and ignorance is even considered a benchmark for achieving the ideal of a democracy \citep{sanders_against_1997}. ``Interpersonal discussions across social divides can help diverse groups of people peacefully identify solutions to shared problems, avoid violent conflict, and come to understand one another better" \citep{argyle_leveraging_2023}. Most of these outcomes require civility of interlocutors, defined as ``a means of demonstrating mutual respect" \citep{mutz_inyourface_2016}. While there was once hope that social media could advance democratic deliberation through open discourse, diverse perspectives, and broad participation, the crucial ingredient of civility is lacking. ``Concern about civil discourse accompanies any technological advance that lowers the cost of information production and distribution" \citep{munger_dont_2021}. 



From the printing press during the Reformation (Bejan 2017) to the invention of cable news with its personalized news reporting \citep{mutz_inyourface_2016,berry2014outrage}, history is filled with examples of concerns about incivility of discourse. ``The 2016 US presidential election took place during a time of rapidly changing civility \citep{munger_dont_2021}, these uncivil exchanges were experienced ``first-hand" through broadcasting and online 
\citep{mutz_inyourface_2016}, with an estimated 68\% of US adults on a social network, up from 25\% in 2008 \citep{duggan_social_2016}, and the hyper-competitve environment of 2016 election politics.




%By highlighting the surveying the dimensions of what constitutes perceptions of a civil discourse, our research informs the literature on content moderation \citep{argyle_leveraging_2023}.


% Social scientists have long observed that “conversation is the soul of democracy” (1, 2). Interpersonal discussions across social divides can help diverse groups of people peacefully identify solutions to shared problems, avoid violent conflict, and come to understand one another better (2–8). Historically, these conversations have occurred face-to-face (8), but online conversations now play a central role in public dialogue. More than 100 billion messages are sent every day on Facebook and Instagram alone (9), and approximately 7 billion conversations occur daily on Facebook Messenger (10). Such conversations can have far-reaching impact. Some of the largest social movements in human history have emerged out of sprawling conversations on social media, and discussions between high-profile social media users can shape the stock market, politics, and many other aspects of human experience (11–14). The internet thus has the capacity to empower an ever-increasing number of people to communicate and deliberate together

\section{Data}

4,675 survey respondents evaluated a total of 9,994 comments from Facebook conversations in terms of toxicity, perceived hostility, disagreement, intolerance, and other characteristics. 

The data is cross-classified. Each respondent was given a random sample of seven comments to evaluate, and not all comments were reviewed by all respondents. On average, each comment was evaluated 3.27 times. 

We have constructed a unique multilevel data set of 1,058 online foci (i.e., public pages) on Facebook during the 2016 US presidential election campaign from 2015 to 2017 February. In order to secure enough variation across the whole ideological spectrum on Facebook, we first obtained a list of the 500 most active political pages of the platform as identified by \citep{bakshy_exposure_2015} in an article published in Science, as well as 37 pages identified as representative online and offline news sources by Pew Research Center’s American Trends Panel in 2014. These pages make up our media samples. For our politician samples, we obtain a list of all legislators in both the House and Senate from the 113rd to the 115th Congresses. In total, we identified 476 Facebook media pages and 582 legislators’ official Facebook pages that were active in March 2017 and subsequently collected all public posts, public comments, and user-reactions in these forums using the official Facebook API v2.7 (see Data Management Plan). Our forum data has a hierarchical structure: each page (i) includes multiple posts (ii), which in turn contain comments (iii) and their subcomments (iv). Across all of these levels, users are able to react or comment on the material other users have posted. The main online foci we focus on in this project are the pages, as they tend to have stronger and more readily identifiable identities than posts or comments.

\section{Methods}

\newpage
\section{Results}



\begin{table}[!htbp] \centering 
  \caption{} 
  \label{} 
\small 
\begin{tabular}{@{\extracolsep{-11pt}}lD{.}{.}{-2} D{.}{.}{-2} D{.}{.}{-2} D{.}{.}{-2} } 
\\[-1.8ex]\hline 
\hline \\[-1.8ex] 
 & \multicolumn{4}{c}{Dependent variable} \\ 
\cline{2-5} 
\\[-1.8ex] & \multicolumn{4}{c}{BToxicNum01} \\ 
\\[-1.8ex] & \multicolumn{1}{c}{(1)} & \multicolumn{1}{c}{(2)} & \multicolumn{1}{c}{(3)} & \multicolumn{1}{c}{(4)}\\ 
\hline \\[-1.8ex] 
 PolIdComp2Sd &  & -0.06^{***} & -0.06^{***} & -0.06^{***} \\ 
  Age2Sd &  &  & -0.03^{***} & -0.03^{***} \\ 
  GenderOther &  &  & -0.01 & -0.01 \\ 
  GenderWoman &  &  & 0.001 & 0.0001 \\ 
  EducationNum2Sd &  &  & 0.02^{***} & 0.02^{***} \\ 
  MaritalStatusMarried &  &  & 0.0002 & -0.0001 \\ 
  MaritalStatusNever married &  &  & 0.002 & 0.002 \\ 
  MaritalStatusPrefer not to answer &  &  & -0.01 & -0.01 \\ 
  MaritalStatusSeparated &  &  & 0.04 & 0.04 \\ 
  MaritalStatusWidowed &  &  & -0.01 & -0.01 \\ 
  ReligionEvangelical Protestant &  &  & -0.01 & -0.01 \\ 
  ReligionJewish &  &  & -0.02 & -0.02 \\ 
  ReligionMainline Protestant &  &  & -0.02^{**} & -0.02^{**} \\ 
  ReligionMormon &  &  & -0.02 & -0.02 \\ 
  ReligionMuslim &  &  & 0.03 & 0.03 \\ 
  ReligionNot available &  &  & -0.01 & -0.01 \\ 
  ReligionNot religious &  &  & -0.01 & -0.01 \\ 
  ReligionOther Religion &  &  & 0.001 & -0.0002 \\ 
  SexualOrientationGay or lesbian &  &  & -0.001 & -0.001 \\ 
  SexualOrientationHeterosexual / straight &  &  & -0.01 & -0.01 \\ 
  SexualOrientationOther &  &  & 0.02 & 0.02 \\ 
  HhSizeHh2 &  &  & 0.004 & 0.004 \\ 
  HhSizeHh3 &  &  & 0.01 & 0.01 \\ 
  HhSizeHh4 &  &  & 0.01 & 0.01 \\ 
  HhSizeHh5 &  &  & 0.002 & 0.003 \\ 
  HhSizeHh6 or more &  &  & 0.01 & 0.01 \\ 
  HhSizeHhOther &  &  & 0.003 & 0.004 \\ 
  RegionNortheast &  &  & -0.01^{*} & -0.01^{*} \\ 
  RegionSouth &  &  & -0.004 & -0.004 \\ 
  RegionWest &  &  & -0.003 & -0.003 \\ 
  Income2Sd &  &  & -0.01 & -0.01 \\ 
  RaceBlack &  &  & 0.01 & 0.01 \\ 
  RaceHispanic &  &  & 0.03^{**} & 0.03^{**} \\ 
  RaceOther &  &  & -0.01 & -0.01 \\ 
  RaceWhite &  &  & 0.02^{**} & 0.02^{**} \\ 
  Order &  &  & -0.01^{***} & -0.01^{***} \\ 
  anticonservative &  &  &  & 0.10^{***} \\ 
  america &  &  &  & 0.02^{***} \\ 
  christianity &  &  &  & -0.04 \\ 
  suggestive &  &  &  & 0.18^{***} \\ 
  drugs &  &  &  & -0.06^{***} \\ 
  ideo\_commenterB &  &  &  & 0.07^{***} \\ 
  target\_likes\_count &  &  &  & -0.0003^{**} \\ 
  Constant & -0.0000 & 0.0000 & -0.01 & -0.08^{***} \\ 
 \hline \\[-1.8ex] 
Observations & \multicolumn{1}{c}{32,695} & \multicolumn{1}{c}{32,695} & \multicolumn{1}{c}{32,695} & \multicolumn{1}{c}{32,695} \\ 
Log Likelihood & \multicolumn{1}{c}{-11,041.93} & \multicolumn{1}{c}{-10,956.70} & \multicolumn{1}{c}{-10,774.14} & \multicolumn{1}{c}{-10,137.60} \\ 
Akaike Inf. Crit. & \multicolumn{1}{c}{22,091.85} & \multicolumn{1}{c}{21,923.40} & \multicolumn{1}{c}{21,628.28} & \multicolumn{1}{c}{20,369.19} \\ 
Bayesian Inf. Crit. & \multicolumn{1}{c}{22,125.43} & \multicolumn{1}{c}{21,965.38} & \multicolumn{1}{c}{21,964.08} & \multicolumn{1}{c}{20,763.76} \\ 
\hline 
\hline \\[-1.8ex] 
\textit{Note:}  & \multicolumn{4}{r}{$^{*}$p$<$0.1; $^{**}$p$<$0.05; $^{***}$p$<$0.01} \\ 
\end{tabular} 
\end{table} 





\newpage
\bibliographystyle{asr}
\bibliography{bib}

\section{Supplementary Material}

\end{document}
